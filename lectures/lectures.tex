\documentclass[10pt, a4paper]{article}

% DRAFT
%\usepackage[light,first,bottomafter]{draftcopy}

%\usepackage{changebar}

% slovencina
\usepackage[activeacute, slovak, english]{babel}
%\noextrasslovak
\usepackage[utf8]{inputenc}
\usepackage[T1]{fontenc}

% rozmery stranky
%\usepackage{a4wide}
\addtolength{\voffset}{-3cm}
\addtolength{\hoffset}{-2cm}
\addtolength{\textwidth}{4 cm}
\addtolength{\textheight}{5 cm}
\usepackage{array}

\linespread{1.2}

% fonty pre text a matematiku
%\usepackage{newcent}
%\usepackage{euler}
\usepackage{amsmath}


% AMS-TeX
\usepackage{amsmath}
\usepackage{amsfonts}
\usepackage{amssymb}
\usepackage{multicol}
\usepackage{amsthm}

\newcommand*\ruleline[1]{\par\noindent\raisebox{.8ex}{\makebox[\linewidth]{\hrulefill\hspace{1ex}\raisebox{-.8ex}{#1}\hspace{1ex}\hrulefill}}}

\def\ans#1{\big[\hskip 2mm {#1}\hskip 2mm\big]}
\def\ogf{\overset{{\rm ogf}}{\longleftrightarrow}}
\def\xD{x{\rm D}}
\def\D{{\rm D}}

\begin{document}
\selectlanguage{slovak}

\addtolength{\parskip}{0.5\baselineskip}

%\pagestyle{empty}


\section{Basics of generating functions}
\begin{itemize}
\item 
Introduction [Wilf 1--3]:
\begin{itemize}
    %\setlength\itemsep{.2em}
    \item how to define a sequence: exact formula, recurrent relation (Fibonacci), algorithm (the sequence of primes); there are uncomputable sequences (programs that do not stop)
    \item a new way: power series (members of the sequence as coefficients in the series)
    \item advantages: many advanced tools from analytical theory of functions
    \item very powerful: works on many sequences where nothing else is known to work
    \item allows to get asymptotic formulas and statistical properties
    \item powerful way to prove combinatorial identities
    \item ``Konečne vidím, že je tá matalýza aj na niečo dobrá. Keby mi to bol niekto predtým povedal\dots''
\end{itemize}

\item 
Two examples [Wilf 3--7]:
\begin{itemize}
    \item $a_{n+1} = 2a_n + 1$ for $n\ge 0$, $a_0 = 0$
    \item write few members, guess $a_n = 2^n-1$, provable by induction
    \item multiply by $x^n$, sum over all $n$, assign gf: $\displaystyle \qquad{A(x)\over x}=2A(x)+{1\over 1-x}$
    \item partial fraction expansion: $\displaystyle \qquad A(x)={x\over (1-x)(1-2x)}={1\over 1-2x}-{1\over 1-x}$
    \item the method stays basically the same for harder problems
    \item $a_{n+1}=2a_n+n$ for $n\ge 0$, $a_0=1$
    \item exact formula not obvious; no unqualified variables in the recurrence
    \item obstacle: $\sum_{n\ge 0} nx^n = x/(1-x)^2$; solution: differentiation
    \item concern: is differentiation allowed? discussed later, but in principle yes:
        in formal power series (as an algebraic ring) or via convergence (if we care about analytical properties)
    \item $\displaystyle A(x) = {1-2x+2x^2\over (1-x)^2(1-2x)} = {A\over (1-x)^2} + {B\over 1-x} + {C\over 1-2x} = {-1\over (1-x)^2} + {2\over 1-2x}$
    \item $1/(1-x)^2$ is just $x/(1-x)^2$ (see above) shifted by $1$
    \item $a_n=2^{n+1}-n-1$
\end{itemize}

\item 
The method [Wilf 8]:
\begin{itemize}
    \item 1. make sure variables in the recurrence are qualified (e.g. range for $n$)
    \item 2. name and define the gf
    \item 3. multiply by $x^n$, sum over all $n$ in the range
    \item 4. express both sides in terms of the gf
    \item 5. solve the equation for gf
    \item 6. calculate coefficients of gf power series expansion
    \item useful notation: $[x^n]f(x)$; e.g. $$[x^n]e^x=1/n!\qquad [t^r]{1\over 1-3t}=3^r\qquad [v^m](1+v)^s={s\choose m}$$         
\end{itemize}

\item
Solve $a_n=5a_{n-1}-6a_{n-2}$ for $n\ge 2$, $a_0 = 0$, $a_1=1$. \ans{$G(x) = {x\over (1-2x)(1-3x)}$; $a_n = 3^n-2^n$}

\item 
Fibonacci [Wilf 8--10]:
\begin{itemize}
    \item three-term recurrence: $F_{n+1}=F_n+F_{n-1}$ for $n\ge 1$, $F_0=0$, $F_1=1$.
    \item apply the method ($r_\pm = (1\pm \sqrt 5)/2$): $$\displaystyle F(x) = {x\over 1-x-x^2} = {x\over (1-xr_{+})(1-xr_{-})}={1\over r_{+}-r_{-}}\left({1\over 1-xr_{+}}-{1\over 1-xr_{-}}\right)$$
    \item $F_n={1\over \sqrt 5}(r_{+}^n-r_{-}^n)$
    \item the second term is ${} < 1$ and goes to zero, so the first term ${1\over \sqrt 5}({1+\sqrt 5\over 2})^n$ gives a good approximation
\end{itemize}
    
\item
Find ogf for the following sequences (always $n\ge 0$) [W1.1]:

\setlength\extrarowheight{1mm}
\begin{tabular}{cl@{\hskip 5mm}l}
    (a) & $a_n = n$ & \ans{introduce $\xD$; $(\xD){1\over 1-x} = {x\over (1-x)^2}$}\\
    (b) & $a_n = \alpha n + \beta$ & \ans{$\alpha x/(1-x)^2+\beta/(1-x)$}\\
    (c) & $a_n = n^2$ & \ans{$(\xD)^2 1/(1-x) = {1+x\over (1-x)^3}$}\\
    (d) & $a_n = n^3$ & \ans{$(\xD)^3 1/(1-x)$}\\
    (e) & $a_n = P(n)$; $P$ is a polynomial of degree $m$ & \ans{$P(\xD){1\over 1-x}$}\\
    (f) & $a_n = 3^n$ & \ans{$1/(1-3x)$}\\
    (g) & $a_n = 5\cdot 7^n-3\cdot 4^n$ & \ans{${5\over (1-7x)}-{3\over 1-4x}$}\\
    (h) & $a_n = (-1)^n$ & \ans{$1/(1+x)$}\\
\end{tabular}

\item 
Find the following coefficients [W1.5]:
\begin{center}
\setlength\extrarowheight{1mm}
\begin{tabular}{cl@{\hskip 5mm}l}
    (a) & $[x^n]\, e^{2x}$ & \ans{$2^n/n!$}\\

    (b) & $[x^n/n!]\, e^{\alpha x}$ & \ans{$\alpha^n$}\\
    (c) & $[x^n/n!]\, \sin x$ & \ans{$(-1)^m$ if $n=2m+1$ is odd, $0$ otherwise}\\
    (d) & $[x^n]\, 1/(1-ax)(1-bx)$ ($a\neq b$) & \ans{$(a^{n+1}-b^{n+1})/(a-b)$}\\
    (e) & $[x^n]\, (1+x^2)^m$ & \ans{$[2\mid n]{m\choose n/2}$}\\
\end{tabular}
\end{center}

\item
Compute $\square_n = \sum_{k=1}^n k^2$.
\begin{itemize}
    \item assign ogf to the sequence $1^2, 2^2, \dots, n^2$: $f(x) = \sum_{k=1}^n{k^2x^k}$
    \item $(\xD)^2 [(x^{n+1}-1)/(x-1)] = x {-2 n^2 x^{n + 1} + n^2 x^{n + 2} + n^2 x^n - 2 n x^{n + 1} + x^{n + 1} + 2 n x^n + x^n - x - 1)\over (x - 1)^3}$
    \item note that $\square_n = f(1) = \lim_{x\to 1} (xD)^2 [(x^{n+1}-1)/(x-1)]=n(n+1)(2n+1)/6$
\end{itemize}

\item
Find the sequence with gf $1/(1-x)^3$.

\item
Find a linear recurrence going back two sequence members that has a solution that contains $n\cdot 3^n$ (possibly plus some linear combination of other exponential or polynomial factors).

\item
Find explicit formulas for the following sequences [W1.6, R2, R3, R7]:

\setlength\extrarowheight{1mm}
\begin{tabular}{cl@{\hskip 3mm}l}
    (a) & $a_{n+1} = 3a_n+2$ for $n\ge 0$; $a_0=0$ & \ans{$3x/(1-x)(1-3x)$; \quad $3^n-1$}\\
    (b) & $a_{n+1} = \alpha a_n + \beta$ for $n\ge 0$; $a_0=0$ & \ans{$\beta x/(1-x)(1-\alpha x)$;\quad ${\alpha^n-1\over \alpha-1}\beta$}\\
    (c) & $a_{n+1} = a_n/3  +1$ for $n\ge 0$; $a_0=1$ & \ans{${3/2\over 1-x}-{1/2\over 1-x/3}$;\quad ${3^{n+1}-1\over 2\cdot 3^n}$}\\
    (d) & $a_{n+2} = 2a_{n+1}-a_n$ for $n\ge 0$, $a_0=0$, $a_1=1$ & \ans{$x/(1-x)^2$;\quad $n$}\\
    (e) & $a_{n+2} = 3a_{n+1}-2a_n+3$ for $n\ge 0$; $a_0=1$, $a_1=2$ & \ans{${4\over 1-2x}-{3\over (1-x)^2}$;\quad $2^{n+2}-3n-3$}\\
    (f) & $a_n = 2a_{n-1}-a_{n-2}+(-1)^n$ for $n>1$; $a_0=a_1=1$ & \ans{${1/2\over (1-x)^2}-{1/4\over 1-x}+{1/4\over 1+x}$; ${2n+3+(-1)^n\over 4}$}\\
    (g) & $a_n = 2a_{n-1}-n\cdot(-1)^n$ for $n\ge 1$; $a_0=0$ & \ans{${x/9-2/9\over (1+x)^2}+{2/9\over 1-2x}$; ${2^{n+1}-(3n+2)(-1)^n\over 9}$}\\
    (h) & $a_n = 3a_{n-1} + {n\choose 2}$ for $n\ge 1$; $a_0=2$ & \ans{${1\over 8}(19\cdot 3^n-2n(n+2)-3)$}\\
    (i) & $a_n = 2a_{n-1}-a_{n-2}-2$ for $n\ge 2$; $a_0=a_{10}=0$ & \ans{$n(a_1+1-n)$, so with $a_{10}$, $a_n=n(10-n)$}\\
    (j) & $a_n = 4(a_{n-1}-a_{n-2})+(-1)^n$ for $n\ge 2$; $a_0=1$, $a_1=4$ & \ans{${1+x+x^2\over (1+x)(1-2x)^2}$; ${(-1)^n\over 9}-{5\over 18}\cdot 2^n+{7\over 6}(n+1)2^n$}\\
    (k) & $a_n = -3a_{n-1}+a_{n-2}+3a_{n-3}$ for $n\ge 3$; $a_0=20$, $a_1=-36$, $a_2=60$ \hspace*{-2cm} & \hspace*{2cm} \ans{$5(-3)^n+18(-1)^n-3$}\\
\end{tabular}

\end{itemize}

\newpage



\section{Ordinary generating functions}
% 2. prednaska

\begin{itemize}

\item From the homework: solve $a_n = 2a_{n-1}-a_{n-2}-2$ for $n\ge 1$; $a_0=a_{10}=0$.\\
Applying the standard method, while keeping $a_1$ as a parameter, we get
$$
A(x)={a_1x-a_1x^2-2x^2\over (1-x)^3}={a_1x\over (1-x)^2}+{x(1-x)\over (1-x)^3}-{x^2+x\over (1-x)^3},
$$
so $a_n=(a_1+1)n-n^2$. From $a_{10}=0$ we get $a_1=9$, thus $a_n=n(10-n)$.

\item 
Another way for boundary problems (this particular example is motivated by splines, Wilf 10--11):
\begin{itemize}
    \item consider $au_{n+1}+bu_n+cu_{n-1}=d_n$ for $1\le n\le N-1$; $u_0=u_N=0$.
    \item similar to Fibonacci with two given non-consecutive terms (but more general)
    \item define $U(x)= \sum_{j=0}^N u_jx^j$ (unknown); $D(x)=\sum_{j=1}^{N-1} d_jx^j$ (known)
    \item derive $\displaystyle a\cdot {U(x)-u_1x\over x}+bU(x)+cx(U(x)-u_{N-1}x^{N-1}) = D(x)$
    \item $(a+bx+cx^2) U(x) = x D(x)  +au_1x + cu_{N-1}x^N$ (*)
    \item plug in suitable values of $x$ (roots $r_{+}$ and $r_{-}$ of the quadratic polynomial on the LHS)
    \item solve the system of two linear equations and two uknowns $u_1$, $u_{N-1}$
    \item if the roots are equal, differentiate (*) to obtain the second equation
\end{itemize}

\item
Mutually recursive sequences [Knuth 343, Example 3]

\begin{itemize}
    \item consider the number $u_n$ of tilings of $3\times n$ board with $2\times 1$ dominoes
    \item define $v_n$ as the number of tilings of $3\times n$ board without a corner
    \item $u_n = 2v_{n-1} + u_{n-2}$; \quad $u_0 = 1$; $u_1 = 0$
    \item $v_n = v_{n-2} + u_{n-1}$; \quad $v_0 = 0$; $v_1 = 1$
    \item derive $$U(x) = {1-x^2\over 1-4x^2+x^4},\qquad V(x) = {x\over 1-4x^2+x^4}$$
    \item consider $W(z) = 1/(1-4z+z^2)$; $U(x) = (1-x^2)W(x^2)$, so $u_{2n} = w_n - w_{n-1}$

    \item hence $u_{2n} = {(2+\sqrt 3)^n\over 3-\sqrt 3} + {(2-\sqrt 3)^n\over 3+\sqrt 3} = \left\lceil {(2+\sqrt 3)^n\over 3-\sqrt 3}\right\rceil$
            (derivation as a homework)
\end{itemize}

\item
Given $f(x)\ogf (a_n)_{n\ge 0}$, express ogf for the following sequences in terms of $f$ [W1.3]:\\
\setlength\extrarowheight{1mm}
\begin{tabular}{cl@{\hskip 5mm}l}
    (a) & $(a_n+c)_{n\ge 0}$ & \ans{$f(x)+c/(1-x)$}\\
    (b) & $(na_n)_{n\ge 0}$ & \ans{$\xD f(x)$}; \qquad napísať im $(P(n)a_n)_{n\ge 0} \longleftrightarrow P(\xD)f(x)$\\
    (c) & $0, a_1, a_2, a_3, \dots$ & \ans{$f(x)-a_0$}\\
    (d) & $0, 0, 1, a_3, a_4, a_5,\dots$ & \ans{$f(x)-a_0-a_1x+(1-a_2)x^2$}\\
    (e) & $(a_{n+2}+3a_{n+1}+a_n)_{n\ge 0}$ & \ans{$(f-a_0-a_1x)/x^2 + 3(f-a_0)/x + f$}\\
    (f) & $a_0, 0, a_2, 0, a_4, 0, a_6, 0\dots$ & \ans{$(f(x)+f(-x))/2$}\\
    (g) & $a_0, 0, a_1, 0, a_2, 0, a_3, 0\dots$ & \ans{$f(x^2)$}\\
    (h) & $a_1, a_2, a_3, a_4,\dots$ & \ans{$(f(x)-a_0)/x$}\\
    (i) & $a_0, a_2, a_4, \dots$ & \ans{$(f(\sqrt{x})+f(-\sqrt{x}))/2$}\\
\end{tabular}

\end{itemize}

\newpage

\ruleline{GFs of two variables}
\begin{itemize}
    \item introducing a new variable and changing the order of summation can help
    \begin{eqnarray}
        \sum_{n\ge 0} {n\choose k}x^n &=& [y^k]\sum_{m\ge 0} \left(\sum_{n\ge 0} {n\choose m}x^n\right)y^m = [y^k]\sum_{n\ge 0} (1+y)^nx^n\nonumber\\
        &=& [y^k] {1\over 1-x(1+y)} = {1\over 1-x}[y^k] {1\over 1-{x\over 1-x}y} = {x^k\over (1-x)^{k+1}} \label{binomial}
    \end{eqnarray}

    \item alternatively, one can use binomial theorem (Knuth 199/5.56 and 5.57):
    \begin{eqnarray*}
        {1\over (1-z)^{n+1}} &=& (1-z)^{-n-1} =\sum_{k\ge 0} {-n-1\choose k}(-z)^k\\
                             &=& \sum_{k\ge 0} {(-n-1)(-n-2)\dots(-n-k)\over k!}(-z)^k = \sum_{k\ge 0} {n+k\choose n}z^k
    \end{eqnarray*}
\end{itemize}


\ruleline{Formal power series [Wilf chapter 2]}
\begin{itemize}
    \item a ring with addition and multiplication $\sum_n a_nx^n\sum_n b_nx^n = \sum_n \sum_k (a_k b_{n-k})x^n$
    \item if $f(0)\neq 0$, then $f$ has a unique reciprocal $1/f$ such that $f\cdot 1/f = 1$
    \item composition $f(g(x))$ defined iff $g(0) = 0$ or $f$ is a polynomial (cf. $e^{e^x-1}$ vs. $e^{e^x}$)
    \item formal derivative $\D$: $\D\sum_n a_nx^n = \sum na_nx^{n-1}$; usual rules for sum, product etc.
    \item exercise: find all $f$ such that $\D f = f$
\end{itemize}

\ruleline{Rules for manipulation [Wilf 2.1, Knuth 334]. Assume that $f\ogf (a_n)_{n=0}^\infty$.}

\begin{itemize}
    \item {\bf Rule 1}: for a positive integer $h$, $(a_{n+h})\ogf (f-a_0-\dots-a_{h-1}x^{h-1})/x^h$
    \item {\bf Rule 2}: if $P$ is a polynomial, then $P(\xD)f\ogf (P(n)a_n)_{n\ge 0}$
        \begin{itemize}
            \item example: $(n+1)a_{n+1} = 3a_n+1$ for $n\ge 0$, $a_0 = 1$; thus $f' = 3f + 1/(1-x)$
            \item example: $\sum_{n\ge 0} {n^2+4n+5\over n!}$; thus $f=\sum_{n\ge 0} (n^2+4n+5){x^n\over n!} = ((\xD)^2+4\xD+5)e^x = (x^2+5x+5)e^x$\\
                \hspace*{.5 cm} we need $f(1)=11e$; works because the resulting $f$ is analytic in a disk\\
                \hspace*{.5 cm} containing $1$ in the complex plane (that is, it converges to its Taylor series)
        \end{itemize}
    \item {\bf Rule 3}: if $g\ogf (b_n)$, then $fg\ogf (\sum_{k=0}^n a_kb_{n-k})_{n\ge 0}$
        $$ \sum_{k=0}^n (-1)^kk = (-1)^n\sum_{k=0}^n k\cdot (-1)^{n-k} = (-1)^n[x^n]{x\over (1-x)^2}\cdot{1\over 1+x} = {(-1)^n\over 4}\left(2n+1-(-1)^n\right)$$
    \item {\bf Rule 4}: for a positive integer $k$, we have $\displaystyle f^k\ogf \left(\sum_{n_1+n_2+\dots+n_k=n} a_{n_1}a_{n_2}\dots a_{n_k}\right)_{n\ge 0}$
        \begin{itemize}
            \item example: let $p(n,k)$ be the number of ways $n$ can be written as an ordered sum of $k$ nonnegative integers
            \item according to R4, $(p(n,k))_{n\ge 0}\ogf 1/(1-x)^k$, so $p(n,k) = {n+k-1\choose n}$ thanks to \eqref{binomial}
        \end{itemize}
    \item {\bf Rule 5}: $\displaystyle {f\over (1-x)}\ogf \left(\sum_{k=0}^n a_k\right)_{n\ge 0}$\\
        \begin{itemize}
            \item example: $\displaystyle (\square_n)_{n\ge 0}\ogf {1\over 1-x}\cdot (\xD)^2 {1\over 1-x} = {x(1+x)\over (1-x)^4}$, so by \eqref{binomial}, $\square_n = {n+2\choose 3}+{n+1\choose 3}$
        \end{itemize}
\end{itemize}

\ruleline{Exercises}

\begin{enumerate}
    \item Using Rule 5, prove that $F_0+F_1+\dots+F_n=F_{n+2}-1$ for $n\ge 0$ [Wilf 38, example 6].\\
        \ans{Compare gfs of both sides, left is $f/(1-x)$, where $f = x/(1-x-x^2)$, i.e. Fibonacci.}
    \item Solve $g_n=g_{n-1}+g_{n-2}$ for $n\ge 2$, $g_0 = 0$, $g_{10} = 10$.\\
        \ans{$g_n = {g_{10}\over F_{10}}F_n$, try the ``boundary method'' described above, computer necessary}
    \item Solve $a_n = \sum_{k=0}^{n-1}a_k$ for $n > 0$; $a_0 = 1$. [R16]\\
        \ans{$a_n = 2^{n-1}$ for $n \ge 1$}
    \item Solve $f_n=2f_{n-1}+f_{n-2}+f_{n-3}+\dots+f_1+1$ for $n\ge 1$, $f_0 = 0$ [Knuth 349/(7.41)]\\
        \ans{$F(x) = x/(1-3x+x^2)$; $f_n=F_{2n}$}
    \item Solve $g_n = g_{n-1} + 2g_{n-2}+\dots +ng_0$ for $n> 0$, $g_0 = 1$. [K7.7]\\
        \ans{$G(x)=1+x/(1-3x+x^2)$; $g_n=F_{2n} + [n=0]$}
    \item Solve $g_n = \sum_{k=1}^{n-1} {g_k + g_{n-k} + k\over 2}$ for $n\ge 2$, $g_1 = 1$.
    \item Solve $g_n=g_{n-1}+2g_{n-2}+(-1)^n$ for $n\ge 2$, $g_0 = g_1 = 1$. [Knuth 341, example 2]\\
        \ans{$G(x) = {1+x+x^2\over (1-2x)(1+x)^2}$; $g_n = {7\over 9}2^n + {1\over 9}(3n+2)(-1)^n$}
    \item Solve $a_{n+2}=3a_{n+1}-2a_n+n+1$ for $n\ge 0$; $a_0 = a_1 = 1$. [R24]\\
        \ans{$A(z) = {2\over 1-2z}-{1\over (1-z)^3}$; $a_n = 2^{n+1}-{n+2\choose 2}$}
    \item Prove that $\displaystyle \ln {1\over 1-x} = \sum_{n\ge 1} {1\over n} x^n$. \ans{consider $\int {1\over 1-x}$}
\end{enumerate}



\section{Skipping sequence elements, Catalan numbers}


\ruleline{Discovering combinatorial identities via gfs [Knuth 198, Vandermonde and 5.55]}
\begin{itemize}
    \item $(1+x)^r = \sum_{k\ge 0} {r\choose k}x^k$; consider $(1+x)^r(1+x)^s = (1+x)^{r+s}$
    \item comparison of coefficients yields $\sum_{k\ge 0}^n {r\choose k}{s\choose n-k}={r+s\choose n}$ --- Vandermonde
    \item by considering $(1-x)^r(1+x)^r = (1-x^2)^r$, we obtain $$\sum_{k=0}^n {r\choose k}{r\choose n-k}(-1)^k = (-1)^{n/2}{r\choose n/2}[2\mid n]$$
\end{itemize}

\ruleline{Every third binomial coefficient [Wilf 51, example 4]}
\begin{itemize}
    \item why ${1\over 2}(A(x)+A(-x))\ogf a_0, 0, a_2, 0, a_4, \dots$ works: ${1\over 2}(1^n + (-1)^n) = [2\mid n]$
    \item in general, for $\omega$ being $r$-th root of unity, ${1\over r}\sum_{j=0}^{r-1} (\omega^j)^n = {1\over r}\sum_{j=0}^{r-1} e^{2\pi ijn/r} = [r\mid n]$\\
        --- just a geometric progression, or a consequence of $t^r-1=(t-1)(t^{r-1}+\dots+t+1)$
    \item problem: find $S_n = \sum_k (-1)^k{n\choose 3k}$
    \item if we knew $f(x) = \sum_k {n\choose 3k}x^{3k}$, we would have $S_n = f(-1)$
    \item for $A(x) = (1+x)^n$, we have $f(x) = {1\over 3}\big(A(x) + A(x\omega^1) + A(x\omega^2)\big)$ for $\omega=e^{2\pi i/3}$
    \item and so $S_n=f(-1) ={1\over 3}[(1-\omega)^n + (1-\omega^2)^n)] = $ $$ = {1\over 3}\left[\left({3-\sqrt3 i\over 2}\right)^n+\left({3+\sqrt3 i\over 2}\right)^n\right] = 2\cdot 3^{{n\over 2}-1}\cos\left({\pi n\over 6}\right)$$
\end{itemize}

\ruleline{Catalan numbers [Knuth 357, example 4]}
\begin{itemize}
    \item consider the number of possibilities $c_n$ of how to specify the multiplication order of $A_0A_1\dots A_n$ by parentheses; let $C(x)=\sum_{n\ge 0} c_nx^n$
    \item divide possibilities by the place of last multiplication; $c_n = \sum\limits_{k=0}^{n-1} c_kc_{n-1-k}$ for $n > 0$; $c_0=1$
    \item many ways to deal with the recurrence:
            \begin{itemize}
                \item[(1)] shift the recurrence to $c_{n+1} = \sum_{k=0}^n c_kc_{n-k}$ and use Rules 1 and 3; ${C(x)-1\over x} = C(x)^2$
                \item[(2)] RHS as a convolution of $c_n$ with $c_{n-1}$, i.e. $C(x)\cdot xC(x)$
                \item[(3)] RHS as a convolution of $c_n$ with $c_n$ shifted by Rule 1, i.e. $x\cdot C(x)^2$
                \item[(4)] rewriting through sums and changing the order of summation:
                    $$\sum_{n\ge 1}x^n\sum_{k=0}^{n-1}c_kc_{n-1-k}=\sum_{k=0}^\infty x^kc_k\sum_{n\ge k+1} c_{n-1-k}x^{n-k}=
                        \sum_{k=0}^\infty x^kc_k xC(x)=xC(x)\cdot C(x)$$
            \end{itemize}
    \item consequently, $C(x) - 1 = xC(x)^2$ and thus $C(x) = {1\pm \sqrt{1-4x}\over 2x}=\displaystyle {1\over 2x}\left(1 - \sqrt{1-4x}\right)$
    \item we want $C$ continuous and $C(0) = 1$, so we choose the minus sign (note that the resulting function below
            is analytical since ${2n\choose n}/(n+1) < 2^{2n}$; it would be analytical also if we chose the plus sign)
    \item binomial theorem yields
    \begin{eqnarray*}
        \sqrt{1-4x} = (1-4x)^{1/2} = \sum_{k\ge 0} {1/2\choose k}(-4x)^k &=& 1+\sum_{k\ge 1}{1\over 2k\cdot (-4)^{k-1}}{2k-2\choose k-1}(-4)^kx^k\\
        &=& 1 - \sum_{k\ge 1}{2\over k}{2k-2\choose k-1}x^k 
    \end{eqnarray*}
    \item we used ${1/2\choose k}={1/2\over k}{-1/2\choose k-1} = {1\over 2k(-4)^{k-1}}{2k-2\choose k-1}$ because ${-1/2\choose m}={1\over (-4)^m}{2m\choose m}$
    \item therefore, $$C(x)={1\over 2x}\sum_{k\ge 1}{2\over k}{2k-2\choose k-1}x^k = \sum_{n\ge 0}{1\over n+1}{2n\choose n}x^n$$
\end{itemize}

\ruleline{Exercises}
\begin{enumerate}
    \item Assume that $A(x)\ogf (a_n)$. Express the generating function for $\sum_{n\ge 0} a_{3n}x^n$ in terms of $A(x)$.\\
            \ans{${1\over 3}(A(x^{1/3}) + A(\omega x^{1/3})) + A(\omega^2 x^{1/3})$, where $\omega=e^{2\pi i/3}$}
    \item Compute $S_n=\sum_{n\ge 0} F_{3n}/10^n$ (by plugging a suitable value into the generating function for $F_{3n}$).\\
            \ans{The gf is ${2x\over 1-4x-x^2}$ and $S_n=20/59$.}
    \item Compute $\sum_k {n\choose 4k}$. \qquad
            % https://math.stackexchange.com/questions/142260/sum-of-every-kth-binomial-coefficient
            % https://www.wolframalpha.com/input?i=sum+over+k+of+binom%28n%2C+4k%29
            \ans{$2^{{n\over 2} - 2} \left(2^{n\over 2} + \cos\left({1\over 4}n \pi\right) + (-1)^n \cos\left({3\over 4}n \pi\right)\right)$}
    \item Compute $\sum_k {6m\choose 3k+1}$. \qquad
            % https://math.stackexchange.com/questions/142260/sum-of-every-kth-binomial-coefficient
            % https://www.wolframalpha.com/input?i=sum+over+k+of+binom%28100%2C+3k%2B1%29
            \ans{Compute it for general $n$ and then plug in $n=6m$; $(2^{6m}-1)/3$}
    \item Evaluate $S_n = \sum_{k=0}^n (-1)^k k^2$. \qquad
            \ans{$f(x) = {-x\over (1+x)^3}$; $S_n={1\over 2}(-1)^n n(n+1)$}
    \item Find ogf for $H_n = 1 + 1/2 + 1/3 + \dots$. \qquad
            \ans{$-\ln(1-x) / (1-x)$}
    \item Find the number of ways of cutting a convex $n$-gon with labelled vertices into triangles.\\
            % https://math.mit.edu/~goemans/18310S15/generating-function-notes.pdf
            \ans{$C_{n-2}$ (shifted Catalan numbers)}
    % TODO \item samplesort Knuth 354 ex. 2: convolution...
\end{enumerate}

\newpage



\section{Snake Oil}

The Snake Oil method [Wilf 118, chapter 4.3] -- external method vs. internal manipulations within a sum.

\begin{enumerate}
    \item identify the free variable and give the name to the sum, e.g. $f(n)$
    \item let $F(x) = \sum f(n)x^n$
    \item interchange the order of summation; solve the inner sum in closed form
    \item find coefficients of $F(x)$
\end{enumerate}

\begin{itemize}
\item
Example 0
\begin{itemize}
    \item let's evaluate $f(n) = \sum_k {n\choose k}$; after Step 2, $F(x) = \sum_{n\ge 0} x^n \sum_k {n\choose k}$
    \item $\displaystyle F(x) = \sum_k \sum_n {n\choose k}x^n = \sum_k {x^k\over (1-x)^{k+1}}={1\over 1-x}\cdot {1\over 1-{x\over 1-x}}={1\over 1-2x}$
\end{itemize}

\item
Example 1 [Wilf 121]
\begin{itemize}
    \item let's evaluate $f(n) = \sum_{k\ge 0} {k\choose n-k}$
    \item after Step 2, $F(x) = \sum_n x^n \sum_{k\ge 0} {k\choose n-k}$
    \item $\displaystyle F(x) = \sum_{k\ge 0} \sum_n {k\choose n-k}x^n = \sum_{k\ge 0}x^k\sum_n {k\choose n-k}x^{n-k} = \sum_{k\ge 0}x^k (1+x)^k = {1\over 1-x-x^2}$
    \item so $f(n) = F_{n+1}$
\end{itemize}

\item
Example 2 [Wilf 122]
\begin{itemize}
    \item let's evaluate $f(n) = \sum_{k} {n+k\choose m+2k}{2k\choose k}{(-1)^k\over k+1}$, where $m$, $n$ are nonnegative integers
        \begin{eqnarray*}
        F(x) &=& \sum_{n\ge 0} x^n \sum_{k} {n+k\choose m+2k}{2k\choose k}{(-1)^k\over k+1} \\
             &=& \sum_k {2k\choose k}{(-1)^k\over k+1}x^{-k}\sum_{n\ge 0}{n+k\choose m+2k}x^{n+k}\\
             &=& \sum_k {2k\choose k}{(-1)^k\over k+1}x^{-k}{x^{m+2k}\over (1-x)^{m+2k+1}}\\
             &=& {x^m\over (1-x)^{m+1}}\sum_k {2k\choose k}{1\over k+1}\left({-x\over (1-x)^2}\right)^k \\
             &=& {-x^{m-1}\over 2(1-x)^{m-1}}\left(1-\sqrt{1+{4x\over (1-x)^2}}\right) = {x^m\over (1-x)^m}
        \end{eqnarray*}
    \item so $f(n) = {n-1\choose m-1}$
\end{itemize}

\item
Example 6 [Wilf 127]
\begin{itemize}
    \item prove that $\sum_{k} {m\choose k}{n+k\choose m} = \sum_k {m\choose k}{n\choose k}2^k$, where $m$, $n$ are nonnegative integers
    \item the ogf of the left-hand side is
        $$L(x) = \sum_{k} {m\choose k} x^{-k}\sum_{n\ge 0}{n+k\choose m}x^{n+k} ={(1+x)^m\over (1-x)^{m+1}}$$
    \item we get the same for the right-hand side
\end{itemize}

%\item 
%Introducing additional free variable [W4.13]
%\begin{itemize}
%    \item Let's prove that $\sum_k (-1)^{n-k}{2n\choose k}^2={2n\choose n}$.
%    \item We evaluate $\sum_k (-1)^k{n\choose k}{n\choose n-m+k}$ by multiplying by $x^m$ etc. % todo?
%\end{itemize}

\end{itemize}

\newpage

\ruleline{Exercises}
\begin{enumerate}
    \item Prove that $\sum_k k{n\choose k} = n2^{n-1}$ via the snake oil method.
            \ans{$L(x) = P(x) = {x\over (1-2x)^2}$}
    \item Evaluate $\displaystyle f(n)=\sum_k k^2{n\choose k}3^k$. \\[2mm]
            \ans{$F(x)={3x(1+2x)\over (1-4x)^3}={3/8\over 1-4x}-{3/2\over (1-4x)^2}+{9/8\over (1-4x)^3}$; $f(n)=3\cdot 4^{n-2}\cdot n(1+3n)$}
    \item Find a closed form for $\displaystyle \sum_{k\ge 0} {k\choose n-k}t^k$. [W4.11(a)] \\[2mm]
            \ans{$F(x)=1/(1-tx-tx^2)$}
    \item Evaluate $\displaystyle f(n)=\sum_k {n+k\choose 2k}2^{n-k}$, $n\ge 0$. [Wilf 125, Example 4] \\[2mm]
            \ans{$F(x)={1-2x\over (1-x)(1-4x)}={2\over 3(1-4x)}+{1\over 3(1-x)}$; $f(n)=(2^{2n+1}+1)/3$}
    \item Evaluate $\displaystyle f(n)=\sum_{k\le n/2} (-1)^k{n-k\choose k}y^{n-2k}$. [Wilf 122, Example 3]\\[2mm]
            \ans{$F(x)=1/(1-xy+x^2)$}
    \item Evaluate $\displaystyle f(n)=\sum_{k} {2n+1\choose 2p+2k+1}{p+k\choose k}$. [W4.11(c)]\\[2mm]
            \ans{replace $2n+1$ by $m$ and solve for $f(m)={m-p-1\choose p}2^{m-2p-1}$;  $f(2n+1)={2n-p\choose p}4^{n-p}$;\\[2mm]
                    $\displaystyle F(x) = {x\over (1-x)^2}\sum_{k\ge 0} {p+k\choose p} \left({x\over 1-x}\right)^{2(p+k)}
                    ={x^{p+1}\over 2^p}\cdot {(2x)^p\over (1-2x)^{p+1}}$}
    \item Try to prove that $\sum_k {n\choose k}{2n\choose n+k}={3n\choose n}$ via the snake oil method in three different ways: consider the sum
            $$\sum_k {n\choose k}{m\choose r-k}$$
            and the free variable being one of $n$, $m$, $r$.

\end{enumerate}



\newpage


\section{prednaska}
% 5. prednaska


\begin{itemize}
\item
Purpose of asymptotics [Knuth 439]
\begin{itemize}
    \item sometimes we do not have a closed form or it is hard to compare it to other quantities
    \item $\displaystyle S_n = \sum_{k=0}^n {3n\choose k}\sim 2{3n\choose n}$; $\displaystyle S_n = {3n\choose n}\left(2-{4\over n} + O\left({1\over n^2}\right)\right)$
    \item how to compare it with $F_{4n}$? we need to approximate the binomial coefficient
    \item purpose is to find \emph{accurate} and \emph{concise} estimates:\\
            $H_n$ is $\sum_{k\ge 1}^n 1/k$ vs. $O(\log n)$ vs. $\ln n + \gamma + O(n^{-1})$ 
\end{itemize}

\item
Hierarchy of log-exp functions [Hardy, see Knuth 442]
\begin{itemize}
    \item the class $\cal L$ of logarithmico-exponential functions:
            the smallest class that contains constants, identity function $f(n) = n$,
            difference of any two functions from $\cal L$,
            $e^f$ for every $f\in {\cal L}$, $\ln f$ for every $f\in {\cal L}$ that is ``eventually positive''
    \item every such function is identically zero, eventually positive or eventually negative
    \item functions in $\cal L$ form a hierarchy (every two of them are comparable by $\prec$ or $\asymp$)
\end{itemize}

\item
Notations
\begin{itemize}
    \item $f(n) = O(g(n))$ iff $\exists c: |f(n)|\le c|g(n)|$ (alternatively, for $n\ge n_0$ for some $n_0$)
    \item $f(n) = o(g(n))$ iff $\lim_{n\to\infty} f(n)/g(n) = 0$
    \item $f(n) = \Omega(g(n))$ iff $\exists c: |f(n)|\ge c|g(n)|$ (alternatively, \dots)
    \item $f(n) = \Theta(g(n))$ iff $f(n) = O(g(n))$ and $f(n) = \Omega(g(n))$
    \item basic manipulation: $O(f)+O(g) = O(|f|+|g|)$, $O(f)O(g)=O(fg)=fO(g)$ etc.
    \item meaning of $O$ in sums
    \item \emph{relative} vs. \emph{absolute} error
\end{itemize}

\item
Warm-ups
\begin{enumerate}
    \item Prove or disprove: $O(f+g)=f + O(g)$ if $f$ and $g$ are positive. [K9.5] \ans{false}
    \item Multiply $\ln n + \gamma + O(1/n)$ by $n + O(\sqrt n)$. [K9.6] \ans{$n \ln n + \gamma n + O(\sqrt n\ln n)$}
    \item Compare $n^{\ln n}$ with $(\ln n)^n$. \ans{$\prec$}
    \item Compare $n^{\ln\ln\ln n}$ with $(\ln n)!$. \ans{$\prec$}
    \item Prove or disprove: $O(x+y)^2 = O(x^2) + O(y^2)$. [K9.11] \ans{true}
\end{enumerate}

\item
Common tricks
\begin{itemize}
    \item cut off series expansion (works for convergent series, Knuth 451)
    \item substitution, e.g. $\ln(1+2/n^2)$ with precision of $O(n^{-5})$ \ans{${2\over n^2} - {4\over n^4} + O(n^{-6})$}
    \item factoring (pulling the large part out), e.g. ${1\over n^2+n} = {1\over n^2}{1\over 1+{1\over n}}={1\over n^2}-{1\over n^3}+O(n^{-4})$
    \item division, e.g. $\displaystyle {H_n\over \ln (n + 1)}= {\ln n + \gamma + O(n^{-1})\over (\ln n)(1+O(n^{-1}))}=1 + {\gamma\over \ln n} + O(n^{-1})$
    \item exp-log, i.e. $f(x) = e^{\ln f(x)}$
\end{itemize}

\item
Typical situations for approximation
\begin{itemize}
    \item Stirling formula: $\displaystyle n! = \sqrt{2\pi n}\left({n\over e}\right)^n\left(1+{1\over 12n}+{1\over 288n^2}+O(n^{-3})\right)$
    \item harmonic numbers: $H_n = \ln n + \gamma + {1\over 2n} - {1\over 12n^2} + O(n^{-4})$ 
    \item rational functions, e.g. ${n\over n+2} = {1\over 1+{2\over n}} = 1-{2\over n}+{4\over n^2}+O(n^{-3})$
    \item exponentials: $e^{H_n}=ne^\gamma e^{O(1/n)}=ne^\gamma (1+O(1/n))=ne^\gamma + O(1)$
    \item rational function powered to $n$, e.g.
            $$\left(1-{1\over n}\right)^n=e^{n\ln \left(1-{1\over n}\right)}= \exp\left(n\left({-1\over n}+O\left(n^{-2}\right)\right)\right) = e^{-1 + O(n^{-1}))} = {1\over e} + O(n^{-1})$$
    \item binomial coefficient, e.g. $2n\choose n$: factorials and Stirling formula
            \begin{eqnarray*}
                {2n\choose n}={\sqrt{4\pi n}\left({2n\over e}\right)^{2n}(1+O(n^{-1}))\over 2\pi n\left({n\over e}\right)^{2n}(1+O(n^{-1}))^2}=
                        {{2^{2n}}\over \sqrt{\pi n}}(1+O(n^{-1}))
            \end{eqnarray*}
\end{itemize}

\item
Exercises
\begin{enumerate}
    \item Estimate $\ln(1+1/n)+ \ln(1-1/n)$ with abs. error $O(n^{-3})$ \ans{$-1/n^2 + O(n^{-4})$}
    \item Estimate $\ln(2+1/n)- \ln(3-1/n)$ with abs. error $O(n^{-2})$ \ans{$\ln{2\over 3} + {5\over 6n} + O(n^{-2})$}
    \item Estimate $\lg (n-2)$, abs. error $O(n^{-2})$ \ans{${\ln n\over \ln 2}-{2\over n\ln 2} + O(n^{-2})$}
    \item Evaluate $H_n^2$ with abs. error $O(n^{-1})$. \ans{$(\ln n)^2 + 2\gamma \ln n + \gamma^2 + (\ln n)/n + O(1/n)$}
    \item Estimate $n^3/(2+n+n^2)$ with abs. error $O(n^{-3})$ \ans{$n-1-{1\over n}+{3\over n^2}+O(n^{-3})$}
    \item Prove or disprove: [K9.20] (b) $e^{(1+O(1/n))^2} = e + O(1/n)$ \hskip 1 cm (c) $n! = O\left(((1-1/n)^nn)^n\right)$ \ans{yes, no}
    \item Evaluate $(n+2+O(n^{-1}))^n$ with rel. error $O(n^{-1})$. [K9.13] \ans{$n^n\cdot e^2(1+O(n^{-1}))$}

    \item Compare $H_{F_n}$ with $F_{\lceil H_n\rceil}^2$ [K9.2] \ans{$H_{F_n}\sim n\ln\varphi$, $F_{\lceil H_n\rceil}^2= O(n^{\ln\varphi^2})=o(n)$}
    \item Estimate $\sum_{k\ge 0} e^{-k/n}$ with abs. error $O(n^{-1})$. [K9.7] \ans{$n+1/2+O(n^{-1})$}
    \item Estimate $H_n^5/\ln (n + 5)$ with abs. error $O(n^{-2})$. \ans{$2+{\gamma\over \ln n}-{6\over n\ln n}-{3\gamma\over n\ln^2 n}+O(n^{-2})$}
    \item Estimate $2n\choose n$ with relative error $O(n^{-2})$. [A1] \ans{${2^{2n}\over \sqrt{\pi n}}\left(1-{1\over 8n}+O(n^{-2})\right)$}
    \item Estimate $2n+1\choose n$ with relative error $O(n^{-2})$. [A2] \ans{${2^{2n+1}\over \sqrt{\pi n}}\left(1-{1\over 5n}+O(n^{-2})\right)$}
    \item Compare $(n!)!$ with $((n-1)!)!\cdot (n-1)!^{n!}$. [K9.2c] (Homework if not enough time is left.)
\end{enumerate}

\end{itemize}


\section{prednaska}
% 6. prednaska

\newcommand*\dif{\mathop{}\!\mathrm{d}}

\begin{itemize}
\item Warm-ups
\begin{enumerate}
    \item Let $f(n) = \sum_{k=1}^n \sqrt k$. Show that $f(n) = \Theta(n^{3/2})$. Find $g(n)$ such that $f(n) = g(n) + O(\sqrt n)$.
            \ans{$\int_0^n \sqrt x\dif x \le S_n\le \int_1^{n+1} \sqrt x\dif x$; $g(n) = {2\over 3}n\sqrt n$}
    \item Estimate $(n-2)!/(n-1)$ with abs. error $O(n^{-2})$. \ans{TODO consider ${n!\over n(n-1)^2}$}
    \item For a constant integer $k$, estimate $n^{\underline{k}}/n^k$ with abs. error $O(n^{-3})$. [A5]\\[2mm]
            \ans{$\displaystyle 1-{k\choose 2}{1\over n}+{3k^4-10k^3+9k^2-2k\over 24}{1\over n^2}+O\left({1\over n^3}\right)$}
\end{enumerate}

\item Find a good estimate of $P_n = {(2n-1)!!\over n!}$.
\begin{itemize}
    \item obviously $\displaystyle 1.5^{n-1}\le {1\over 1}\cdot {3\over 2}\cdot {5\over 3}\cdot \dots \cdot {(2n-1)\over n}\le 2^{n-1}$
    \item we split the product into a ``small'' part (first $k$ terms, each at least $3/2$ except the first one) and a ``large'' part (remaining $n-k$ terms); then\\
        $P_n\ge \left({2k+1\over k+1}\right)^{n-k}\cdot 1.5^{k-1} = Q_n\cdot 1.5^{k-1}$; we estimate $Q_n$
    \item if we try $k = \alpha n$, then
        $$Q_n = 2^{n-\alpha n} \exp \left((n-\alpha n)\ln \left(1-{1\over 2(\alpha n + 1)}\right)\right)=2^{n(1-\alpha)}e^{{\alpha-1\over 2\alpha}}(1+O(n^{-1})),$$
        so $P_n \ge (2^{1-\alpha}\cdot 1.5^\alpha)^n \Theta(1)$
    \item if we try $k = \ln n$, then
        $$Q_n = \exp\left((n-\ln n)\left[\ln 2 + \ln \left(1-{1\over 2(1+\ln n)}\right)\right]\right);$$
        if we expand $\ln$ into Taylor series, the error will be $1/\ln^k n = \omega(n^{-1})$, so we can get relative error $O(1)$ at best;\\
        anyway, if we carry it through, we get $P_n = \Omega(2^n n^{-c} e^{-0.5n/\ln n})$
    \item if we try $k = \sqrt n$, then
        \begin{align*}
        Q_n &= \exp\left((n-\sqrt n)\left[\ln 2 + \ln \left(1-{1\over 2(1+\sqrt n)}\right)\right]\right)\\
            &= 2^{n-\sqrt n}\exp\left((n-\sqrt n)\left[{-1\over 2\sqrt n} + {3\over 8n}-{7\over 24n^{3/2}}+O(n^{-2})\right]\right)\\
            &= 2^{n-\sqrt n}\exp\left(-{\sqrt n\over 2} + {7\over 8}-{2\over 3\sqrt n}+O(n^{-1})\right),
        \end{align*}
        thus $P_n \ge 2^n \cdot 0.75^{\sqrt n}\cdot e^{{-\sqrt n\over 2}+{7\over 8}-{2\over 3\sqrt n}} (1+O(n^{-1})) = \Omega\left(2^n c^{\sqrt n}\right)$ for $c\in (0, 1)$.
    \item TODO compare with previous estimate from $k=\ln n$; which is better?
    \item another approach: $P_n = {(2n)!\over n! 2^n n!} = {2n\choose n}/2^n = {2^n\over \sqrt{\pi n}}(1+O(n^{-1}))$
\end{itemize}

\end{itemize}

\newpage

\section{prednaska}
% 7. prednaska

\begin{itemize}

\item Estimate $S_n = \sum_{k=1}^n {1\over n^2+k}$ with absolute error (a) $O(n^{-3})$, (b) $O(n^{-7})$. [Knuth 458/Problem 4]
First approach: ${1\over n^2+k}={1\over n^2(1+k/n^2)}$ etc.; second approach: $S_n = H_{n^2+n}-H_n$. (DU)

\item Sums --- gross bound on the tail: $S_n = \sum_{0\le k\le n} k! = n!\left(1+{1\over n}+{1\over n(n-1)}+ \dots\right)$,
all the terms except the first two are at most $1/n(n+1)$, so $S_n = n!(1+{1\over n}+n{1\over n(n-1)}) = n!(1+O(n^{-1}))$

\item Sums --- make the tail infinite:
\begin{align*}
n!\sum_{k=0}^n{(-1)^k\over k!} &= n!\left(\sum_{k=0}^\infty{(-1)^k\over k!}-\sum_{k\ge n+1}{(-1)^k\over k!}\right)\\
                               &= n!\left(e^{-1}-O\left({1\over (n+1)!}\right)\right)= {n!\over e}+O(n^{-1})
\end{align*}

\item Estimate $S_n=\sum_{k=0}^n {3n\choose k}$ with relative error $O(n^{-2})$. We split the sum into a ``small'' and a ``large'' part at $b$ (which is yet to be determined).
\begin{eqnarray*}
\sum_{k=0}^{n} \binom {3n}k&=&\sum_{k=0}^{n} \binom {3n}{n-k}=\sum_{0\leq k<b} \binom {3n}{n-k}+\sum_{b\le k\le n} \binom {3n}{n-k}.\\
\binom{3n}{n-k} &=& \binom{3n}{n} {n(n-1)\cdot\ldots \cdot 1\over (2n+1)(2n+2)\ldots(2n+k)} =\\
                &=& \binom{3n}{n}\cdot\frac{n^k}{(2n)^k}\frac{\prod_{j=0}^{k-1}\left(1-\frac jn\right)}{\prod_{j=1}^k \left(1+\frac j{2n}\right)}=\binom{3n}{n}\cdot\frac{1}{2^k}\cdot\left[1-\frac{3k^2-k}{4n}+O\left(\frac{k^4}{n^2}\right)\right].\\
\sum_{b\le k\le n} \binom {3n}{n-k}&\le &n\cdot \binom{3n}{n-b}=\binom{3n}{n}\cdot \frac{1}{2^b} O(n)=\binom{3n}{n}\cdot O\left(n^{-2}\right) \hbox{if $\sqrt n\succ b\ge 3\lg n$}.\\
\sum_{0\leq k<3\lg n}\frac{1}{2^k}&=&2-\frac{1}{2^{3\lg n}}=2+O(n^{-3}).\\
-\frac{3}{4n}\sum_{0\leq k<3\lg n}\frac{k^2}{2^k}&=&\frac{-9}{2n}+O(n^{-3}).\\
+\frac{1}{4n}\sum_{0\leq k<3\lg n}\frac{k}{2^k}&=&\frac{1}{2n}+O(n^{-3}).\\
O(n^{-2})\cdot\sum_{0\leq k<3\lg n}\frac{k^4}{2^k}&=&O(n^{-2})\\
\end{eqnarray*}
$$\sum_{k=0}^{n} \binom {3n}k=\binom{3n}{n}\cdot\left[2-\frac{4}{n}+ O(n^{-2})\right]$$

\item Estimate $S_n=\sum_{k=0}^n \binom{4n+1}{k+1}$ with relative error $O(n^{-2})$.
$$\binom{4n+1}{k+1}=\binom{4n}{k+1}+\binom{4n}{k}; $$
$$S_n=\sum_{k=0}^n \binom{4n+1}{k+1}=\sum_{k=0}^n\binom{4n}{k}+ \sum_{k=0}^n\binom{4n}{k+1}=\sum_{k=0}^n\binom{4n}{k}+\sum_{k=1}^{n+1}\binom{4n}{k}; $$
$$S_n=2\sum_{k=0}^n\binom{4n}{k}+\binom{4n}{n+1}-\binom{4n}{0}.$$
$$Q_n=\sum_{k=0}^n\binom{4n}{k}=\sum_{k=0}^n\binom{4n}{n-k};$$
$$\binom{4n}{n-k}=\binom{4n}{n}\cdot\frac{\prod_{j=0}^{k-1}(n-j)}{\prod_{j=1}^{k}(3n+j)}=\binom{4n}{n}\cdot\left(\frac 13\right)^3\cdot\frac{\prod_{j=0}^{k-1}(1-j/n)}{\prod_{j=1}^{k}(1+j/3n)}$$
$$Q_n=\sum_{0\leq k\leq 2\log_3 n}\binom{4n}{n-k}+\sum_{2\log_3 n\leq k<n}\binom{4n}{n-k}$$
$$\sum_{2\log_3 n\leq k<n}\binom{4n}{n-k}=O\left(n\cdot\binom{4n}{n-\lceil 2\log_3 n\rceil} \right)=O\left(\binom{4n}{n}\cdot\frac 1n \right).$$
$$\frac{\prod_{j=0}^{k-1}(1-j/n)}{\prod_{j=1}^{k}(1+j/3n)}=\frac{1-\frac 1n\cdot\sum_{0\leq j<k}j+O\left(\frac{k^4}{n^2}\right)}{1+\frac{1}{3n}\cdot\sum_{1< j\leq k}j+O\left(\frac{k^4}{n^2}\right)} = 1+\frac{2k^2+k}{3n}+O\left(\frac{\log^n}{n^2}\right),$$
$$\sum_{0\leq k\leq 2\log_3 n}\binom{4n}{n-k}=\binom{4n}{n}\cdot\sum_{0\leq k\leq 2\log_3 n}\left( \frac 13\right)^k\cdot[1+\frac{2k^2+k}{3n}+O\left(\frac{\log^n}{n^2}\right)]=$$
$$=\frac 32\cdot \binom{4n}{n} (1+O(n^{-1})).$$
$$\binom{4n}{n+1}= \binom{4n}{n}\cdot\frac{3n}{n+1}=3\cdot\binom{4n}{n}(1+O(n^{-1}));$$
$$S_n=6\cdot\binom{4n}{n}(1+O(n^{-1})).$$

\item How many bits are needed to represent a binary tree with $n$ internal nodes?
\begin{itemize}
    \item we need just the internal vertices to capture the structure; what is the relation between the number of internal vertices and total number of vertices?
    \item imagine labeling the vertices by $1,2,\dots,n$ in such a way that we get a binary search tree (descendants in the left subtree are smaller, in the right subtree are larger);
        by summing over possible roots of the tree we get $t_n = \sum_{i=1}^n t_{i-1} t_{n-i}$; $t_0 = 1$
    \item this is the same as for Catalan numbers, so $t_n = {2n\choose n}{1\over n+1}$
    \item and so we need $\log_2 t_n \sim 2n - 1.5\lg n - 0.5 \lg \pi + O(n^{-1})$ bits
\end{itemize}

    
\end{itemize}



\end{document}




*** What to cover:
I have 4.5 lectures

snake oil method

one practical example of algorithm analysis (e.g. that quicksort median of three average case?)

log exp method for asymptotics (1+1/n)^n = e ^ {n ln(1+1/n)}

sum_{k=0}^n {3n\choose k}, vybrat (3n nad n) pred zatvorku a najst asymptotiku





\item 
Introduction (Wilf 1--3):
\begin{itemize}
    \item
\end{itemize}

\item 
Introduction (Wilf 1--3):
\begin{itemize}
    \item
\end{itemize}

    
    
    
    
    
    
\end{itemize}

\end{document}
