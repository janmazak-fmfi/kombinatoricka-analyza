\documentclass[a4paper, 12pt]{article}

\usepackage[utf8]{inputenc}
\usepackage[T1]{fontenc}
\usepackage[slovak]{babel}

\usepackage{amsmath, amsthm, amssymb}

\addtolength{\voffset}{-3cm}
\addtolength{\hoffset}{-1.3cm}
\addtolength{\textwidth}{2.6 cm}
\addtolength{\textheight}{5cm}

\parskip = 2mm
% \parindent = 0pt

\def\R{\mathbb R}
\def\Z{\mathbb Z}

\title{Kombinatorická analýza, 2. časť}
\date{}
\pagestyle{empty}
\begin{document}

\centerline{Kombinatorická analýza, 2. písomka, opravný termín A}

\begin{enumerate}
\item
Nájdite explicitné vyjadrenie $a_n$, ak $a_0=a_{10}=0$ a
$$
a_n = 2a_{n-1}-a_{n-2}-1\quad\hbox{pre každé $n\ge 1$}.
$$

\item
Nájdite explicitné vyjadrenie $b_n$, ak $b_0=1$ a
$$
b_{n+1} = \sum_{k=0}^n (n-k)b_k\quad\hbox{pre každé $n\ge 0$}.
$$
% R14

%Nájdite explicitné vyjadrenie $b_n$, ak $b_0=0$ a
%$$
%b_{n+1} = 1 + \sum_{k=0}^n 2^{k-n}b_k\quad\hbox{pre každé $n\ge 0$}.
%$$

\item
Nájdite uzavretý tvar pre
$$
\sum_k (-1)^k{m\choose k}{m\choose n-k}.
$$

\item
Odhadnite s relatívnou chybou $O(n^{-2})$ hodnotu
$$
\left({n+1\over n+2}\right)^{n+3}.
$$

\item
Odhadnite s absolútnou chybou $O(n^{-3})$ hodnotu
$$
\sum_{k=0}^n {1\over n^2+2k+1}.
$$
% A4

\end{enumerate}


\end{document}

